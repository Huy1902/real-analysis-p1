\subsection{Introduction}
\subsubsection{Excercise}

\begin{exercise}{1}
    Prove that if $A$ and $B$ are two sets in $\mathcal A$ with $A \subseteq B$, then $m(A) \leq m(B)$. This property is called monotonicity.
\end{exercise}
\begin{solution}
Let $C = B \setminus A$, then by countably additive property of $m$ over countable disjoint set, then
\[m(A) = m(B) - m(C) \leq m(B)\]
where the last inequality is because $m$ is a set function with values in $[0, \infty]$
\end{solution} 
 
\begin{exercise}{2} 
Prove that if there is a set $A$ in the collection $\mathcal A$ for which $m(A) < \infty$, then $m(\emptyset)= 0$
\end{exercise}

\begin{solution}
    \begin{align*}
        m(A) = m(A \cup \emptyset) = m(A) + m(\emptyset)
    \end{align*}
    \begin{equation*}
        m(\emptyset) = 0
    \end{equation*}
\end{solution}

\begin{exercise}{3}
Let $\{E_k\}_k$ be countable collection of sets in $\mathcal A$. Prove that $m(\bigcup_k E_k) \leq \sum_k m(E_k)$
\end{exercise}

\begin{solution}
    Let the set $A_1 = \emptyset$ and $A_i = \bigcup_{k < i} E_k$ with $i > 1$. Further define $X_i = E_i \setminus A_i $, then 
    $\bigcup X_i = \bigcup E_i$ and $\{X_i\}$ are countable disjoint sets. 
    \begin{equation*}
        m(\bigcup_k E_k) = m(\bigcup_k X_k) = \sum_k m(X_k) \leq \sum_k m(E_k)
    \end{equation*}
    the last inequality is due to $X_k \subseteq E_k$.
\end{solution}

\begin{exercise}{4}
    A set function $c$, defined on all subsets of $\mathbb R$, is defined as follows. Define $c(E)$ to be $\infty$ if $E$ has infinitly many members and $c(E)$ to be equal the number of the elements in $E$ if $E$ is finite; define $c(\emptyset) = 0$. Show that $c$ is a countably additive and translation invariant set function. This set function is called the counting measure.
\end{exercise}

\begin{solution}
    Sketch: 
    The union of two finite, disjoint is finite and the elements in this union is equal the total number of element in two sets.
    If one of the two set are infinite, there union is infinite and the measure is infinite by definition. 
    
    Translation operation does not alter the number of elements in a set, so $m$ is invariance under translation.
\end{solution}

