\newcommand{\cantor}{\mathcal{C}}
\newcommand{\Ocal}{\mathcal{O}}
\newcommand{\union}{\cup}
\newcommand{\intersect}{\cap}


\subsection{The Cantor set and The Cantor-Lesbegue function}
\subsubsection{Summary}

\noindent To ease the definition of the Cantor set, define the following set functions 
for any real number $r\in \R$ and any disjoint union of intervals $S = \union_n [a_n, b_n]$:
\begin{itemize}
    \item $r + S = \union_n \left[r+a_n, r+b_n\right]$,
    \item $r\times S = \union_n \left[r\times a_n, r\times b_n\right]$.
\end{itemize}

\begin{defn}[The Cantor Set]
    The Cantor set $\cantor$ can be defined by the following recursive sequence of sets
    \begin{align}
        \cantor_0 &= [0,1] \\
        \cantor_{n+1} &= \frac{\cantor_n}{3} \union \left(\frac{2}{3} + \frac{\cantor_n}{3}\right) \\
        \cantor &= \intersect_{n=1}^{\infty} \cantor_n
    \end{align}
\end{defn}

\noindent The collection $\seq{\cantor_n}_{n=0}^{\infty}$ has the properites:
\begin{enumerate}[label=(\roman*)]
    \item $\seq{\cantor_n}_{n=0}^{\infty}$ is a descending sequence of closed sets.
    \item For each $n$, $\cantor_{n}$ is a disjoint union of $2^n$ intervals, each of them with length $1/3^n$.
\end{enumerate}

\begin{thm}
    The Cantor set $C$ is a closed, uncountable set of measure zero.
\end{thm}


\noindent A real-valued function $f$ that is defined on a set of real numbers is said to be
increasing provided $f(u) \leq f(v) \iff u \leq v$ and said to be strictly increasing, provided
$f(u) < f(v) \forall u < v$.


\begin{defn}[The Cantor function]
    Follow \href{https://www.math.purdue.edu/~torresm/lecture-notes/lebesgue-theory/lebesgue-lecture16.pdf}{this link} 
    for a more rigorous definition (you'll need some function sequence convergence definitions)
\end{defn}

\begin{prop}
    The Cantor-Lesbegue function $\varphi$ is an increasing continous function that maps $[0,1]$ to $[0,1]$.
    Its derivative exists on the openset $\mathcal{O} = [0,1]\setminus \cantor$ and is 0.
\end{prop}

\subsubsection{Excercise}
% Problem groups:
%   39-40
\underline{\textbf{Problems done: 38, 41, 42, 43, 44.}}
\bigskip

% \noindent \phantomsection \textbf{Ex 35.} \label{ex:2_35}


\noindent \phantomsection \textbf{Ex 38.} \label{ex:2_38}
\begin{lem}[Lipschitz image of open, bounded intervals]
    Let $f: [a,b]\to \R$ be Lipschitz with Lipschitz constant $c$, $I \subseteq [a,b]$ be any bounded open interval with length $\epsilon$.
    $m^\ast(f(I))$ does not exceed $c\epsilon$.
\end{lem}
\textbf{Proof.}

Express $I$ as $(x-\epsilon/2, x+\epsilon/2)$, denote $y = f(x)$.

We know that $\forall x_0 \in I, \abs{x_0 - x} < \epsilon/2$.
This implies that $\forall y_0 \in f(I), \abs{y_0 - y} \leq c\abs{x_0 - x} \leq c\epsilon/2$.

As such, $f(I)$ is fully contained in the open interval $(y-c\epsilon/2, y+c\epsilon/2)$ and thus $m^\ast(f(I)) \leq c\epsilon$.


\begin{lem}[Limit point]
    $E$ is a closed set in $\R$. $\seq{x_n}$ is any convergent sequence in $E$.
    The limit of $\seq{x_n}$ is in $E$.
\end{lem}
\textbf{Proof.} We prove by contradiction. Assume $\seq{x_n} \to x \not\in E$, then $x \in \R\setminus E$. 

$E$ is closed, so $\R\setminus E$ is open, so there exists $\epsilon > 0$ satisfying $(x-\epsilon, x+\epsilon) \intersect E = \seq{\emptyset}$.

For such an $\epsilon$, there exists $N\in \N$ s.t.
\[n\geq N \implies \abs{x-x_n} < \epsilon \iff x_n \in (x-\epsilon, x+\epsilon)\]
which is a constradiction ($\seq{x_n}$ is in $E$).

\noindent *This lemma is used to prove the following lemma.

\begin{lem}[Lipschitz image of closed, bounded sets]
   Let $f: [a,b]\to \R$ be Lipschitz. The image of every closed set $E\in [a,b]$ is closed.
\end{lem}
\textbf{Proof.}

Let $y$ be any point of closure of $f(E)$. 
By definition, we can always choose $y_n \in (y-1/n, y+1/n) \intersect f(E) \forall n\in\N$, we use this fact to create a sequence $\seq{y_n}$ in $f(E)$ that converges to $y$.

Let $x_n = f^{-1}(y_n) \forall n$. We now show that $y \in f(E)$.

Firstly, $\seq{x_n}$ is a bounded sequence, by the Bolzano-Weierstrass theorem, there exists a convergent subsequence $\seq{x_{n_k}}$.
This is a convergent sequence in $E$ - a closed set in $\R$, so its limit $x$ is in $E$ (by the Limit point Lemma above).

Secondly, $f$ is continous (Lipschitz functions are continous - trivially provable) so $\seq{x_{n_k}} \to x \implies f(\seq{x_{n_k}}) = y_{n_k} \to f(x)$.

Because limits are unique, $y = f(x)$ and so $y\in E$.

Since $f(E)$ contains all its points of closure, $f(E)$ is closed.

\bigskip

\noindent We use the `Lipschitz image' lemmas, and theorem 11 of chapter 2.4 to solve this excercise.
\begin{itemize}
    \item $f$ map sets of measure zero to sets of measure zero.

    Let $Z \in (a,b)$ be any set of measure zero (extending this to sets of measure zero containing endpoints is trivial)

    % Also let $c > 0$ (at $c=0$, $f$ is a constant, proofs are trivial)

    From `theorem 11', for any $\epsilon > 0$, there exists a disjoint countable union of open intervals $\union_k I_k \subseteq (a,b)$ containing $Z$ and has $ \sum_k m(I_k) < \epsilon$.
    
    This implies $m^\ast(f(Z)) \leq m^\ast(\union_k f(I_k)) \leq c\epsilon$ from the `Lipschitz image of open, bounded intervals' lemma (equality happens at $c=0$)

    As $m^\ast(f(Z)) \leq c\epsilon$ for all $\epsilon > 0$, $m^\ast(f(Z)) = 0$.

    \item $f$ maps $F_\sigma$ sets to $F_\sigma$ sets.
    
    Let $E = \union_k E_k \in [a,b]$ be any $F_\sigma$ set.

    By the `Lipschitz image of closed, bounded sets' lemma, $f(E_k)$ is closed for all $k$, as such $f(E) = \union_k f(E_k)$ is an $F_\sigma$ set.

    \item $f$ maps measurable sets to measurable sets.
    
    Let $M \in [a,b]$ be a measurable set. From `theorem 11', there exists an $F_\sigma$ set $E$ contained by $M$ satisfying $m^\ast(M \setminus E) = 0$.

    We have
    \begin{itemize}
        \item[] $f(E)$ is an $F_\sigma$ set (because $E$ is an $F_\sigma$ set), thus it is measurable.
        \item[] $f(M\setminus E)$ is a set of measure zero (since $m^\ast(M\setminus E) = 0$), which is also measurable.
    \end{itemize}
    Because unions of 2 measurable sets are measurable and $f(M) = f(E) \union f(M\setminus E)$, $f(M)$ is measurable.
\end{itemize}


\noindent \phantomsection \textbf{Ex 41.} \label{ex:2_41}

Pick any point $c \in \cantor$, fix any $\epsilon > 0$, let $I = (c-\epsilon, c+\epsilon)$

We pick any $N\in \N$ s.t. $1/3^N < \epsilon$.

\begin{itemize}
    \item[] At step $N$, the singular inverval containing $c$ is a subset of the neighbour $I$, and as such, there is at least 1 point in $\cantor$ within this neighbour $I$ (specifically, 1 end point of the interval containing $c$).
    \item[] At step $N+1$, this inverval is splitted into 2, there are now 2 intevals contained in $I$, we can see there is at least 3 points in $\cantor$ within the neighbour $I$.
    \item[] At step $N+2$, each inverval is splitted into 2 again, there are now 4 intevals contained in $I$, we can see there is at least 7 points in $\cantor$ within the neighbour $I$.
    \item[] $\cdots$
    \item[] At step $N+k$, there is at least $\sum_{i=0}^{k} 2\times2^i - 1$ points in $\cantor$ within $I$.
\end{itemize}

Since $\seq{\sum_{i=0}^{k} 2\times2^i - 1}_{k=0}^{\infty}$ converges to infinity, there are inifinite points in the Cantor set within any neighbour of any points in the Cantor set.

And because the Cantor set is closed, it is perfect.


\noindent \phantomsection\textbf{Ex 42.} \label{ex:2_42}
Proof by contradiction:

Assume a perfect set $X$ is countable, let $\seq{x_k}_{k=1}^{\infty}$ be any of its enumeration.

Let $F_0$ be any closed and bounded subset of $X$. Define $F_{k+1} = \overline{F_k \setminus \seq{x_{k+1}}}$.

We can observe that $\intersect_{k=0}^\infty F_k$ is the intersection of 
\begin{itemize}
    \item descending ($F_{k+1}\subseteq F_k$, as it has 0 or 1 less element.)
    \item countable ($\seq{x_k}_{k=1}^{\infty}$ is an enumeration of a countable set)
    \item non-empty ($F_k \setminus \seq{x_{k+1}}$ cannot be empty, there are infinite points of $X$ within any $x_{k+1}$'s neighbour because $X$ is perfect.)
    \item and closed sets.
    \item of which $F_0$ is bounded.
\end{itemize}
Yet it's intersection is empty, since $\seq{x_k}_{k=1}^{\infty}$ is an enumeration of a countable set. This contradicts the Nested Set Theorem.


\noindent \phantomsection\textbf{Ex 43.} \label{ex:2_43}
I have proven \hyperref[ex:2_41]{\underline{\textbf{Ex 41.}}} and \hyperref[ex:2_42]{\underline{\textbf{Ex 42.}}}.


\noindent \phantomsection\textbf{Ex 44.} \label{ex:2_44}

We prove by contradiction.

Assumes the Cantor set is \textbf{not} nowhere dense in $\R$ (Assumption 1), 
there exists an open set $\Ocal$ s.t. every open subset $U \intersect \cantor \neq \seq{\emptyset}$.
This is equivalent to $\cantor^c \intersect \Ocal = \seq{\emptyset}$, in other words, $\Ocal \subseteq \cantor$.
We now express $\Ocal$ as a union of disjoint open intervals and pick any such interval $(a,b), b-a > 0$.

We now show that there cannot exist an open interval $(a,b) \subseteq \cantor$ also by contradiction.

Assume, a valid open inveral $(a,b) \subseteq \cantor$ does exist (Assumption 2).
Thanks to how the Cantor set is constructed, this open interval must be a subset of $ C_k \forall k$, 
which implies that the length of $(a,b)$ cannot exceed $C_k$'s total length for any $k$, in other words $b-a < (2/3)^k \forall k$.

But because $\seq{(2/3)^k} \to 0$ as $k\to\infty$, we have $b-a < 0$, which contradicts assumption 2.

As such, there cannot exist $\Ocal \subseteq \cantor$. Assumption 1 has created a contradiction.
