\textbf{Exercise 28:}Suppose $A$ is a non-empty, proper subset of $\mathbf{R}$ that is both open and closed. Then there exists $x \in A$ and $y \in \mathbf{R} \setminus A$. Suppose without loss of generality that $x < y$ and define
\[ E = \{ x \in A : x < y \} \]
Then $E$ is non-empty ($x \in E$) and bounded above (by $y$). The completeness axiom implies that there exists a least upper bound of $E$. Let $x^* = \sup E$ and suppose $x^* \in A$. Since $y \notin A$ and $y$ is an upper bound of $E$, we must have $x^* < y$. Therefore there exists $r > 0$ such that $x^* + r < y$. But since $A$ is open, we can also find $r^* \in (0, r)$ such that $(x^* - r^*, x^* + r^*) \subset A$. But this implies $x^* + \frac{r}{2} \in E$, so $x^*$ is not an upper bound for $E$. This contradicts the definition of $x^*$. Now suppose $x^* \in \mathbf{R} \setminus A$. Since $A$ is closed, $\mathbf{R} \setminus A$ is open. Therefore there exists $r > 0$ such that $(x^* - r, x^* + r) \subset \mathbf{R} \setminus A$. Thus if $x \in A$, $x \le x^* - r$. But this means $x^* - r$ is an upper bound of $E$, contradicting the assumption that $x^*$ is the least upper bound.

The above argument shows that $E$ cannot have a least upper bound, a contradiction of the completeness axiom. We conclude that no non-empty, proper subset of $\mathbf{R}$ that is both open and closed can exist.
\textbf{Exercise 31:}
Suppose $E$ is a set containing only isolated points. For each $x \in E$, define $f(x) = (p,q)$ where $p$ and $q$ are rational numbers such that $p < x < q$ and $(p,q) \cap E = \{x\}$. $f$ defines a one-to-one mapping from $E$ to $\mathbf{Q} \times \mathbf{Q}$. By Corollary 4 and Problem 23, $\mathbf{Q} \times \mathbf{Q}$ is a countable set. This means there exists a one-to-one mapping $g$ from $\mathbf{Q} \times \mathbf{Q}$ onto $\mathbf{N}$. The composition $g \circ f$ defines a one-to-one mapping from $E$ to $\mathbf{N}$ (see Problem 20), which implies $E$ is countable (see Problem 17).
\textbf{Exercise 32:}
(i) Suppose $E$ is open and $x \in E$. Then there exists an $r > 0$ such that the interval $(x-r, x+r)$ is contained in $E$. But this means $x \in \text{int } E$, so $E \subseteq \text{int } E$. Since $\text{int } E \subseteq E$ by definition, $E = \text{int } E$.

Conversely, suppose $E = \text{int } E$. If $x$ is a point in $E$, then $x \in \text{int } E$. But this means there exists an $r > 0$ such that the interval $(x-r, x+r)$ is contained in $E$, so $E$ is open.

(ii) Let $E$ be dense in $\mathbf{R}$ and suppose $x \in \text{int}(\mathbf{R} \setminus E)$. Then there exists $r > 0$ such that $(x-r, x+r) \subseteq
\mathbf{R} \setminus E$. 
But this means there does not exist an element of $E$ between any two numbers in $(x-r, x+r)$, contradicting the assumption that $E$ is a dense set.
We conclude that no such $x$ can be found, so $\text{int}(\mathbf{R} \setminus E) = \emptyset$.

Conversely, suppose $\text{int}(\mathbf{R} \setminus E) = \emptyset$. Let $x$ and $y$ be two real numbers satisfying $x < y$ and suppose $(x, y) \subset \mathbf{R} \setminus E$. Let $z \in (x, y)$ and choose $r \in (0, \min(z-x, y-z))$. Then $(z-r, z+r) \subset (x, y)$, so $(z-r, z+r) \subset \mathbf{R} \setminus E$. But this means $z \in \text{int}(\mathbf{R} \setminus E)$, contradicting the assumption that $\text{int}(\mathbf{R} \setminus E) = \emptyset$. Therefore $(x, y) \not\subset \mathbf{R} \setminus E$, which means there must be an element of $E$ between $z$ and $y$. But since $x$ and $y$ were arbitrary, this means $E$ is dense in $\mathbf{R}$.
\subsubsection{Excercise}