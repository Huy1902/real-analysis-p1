\subsubsection{Excercise}

\textbf{Ex 1.} For $a \neq 0$ and $b \neq 0$, show that \((ab)^{-1} = a^{-1}b^{-1}\)

\begin{align*}
    (a^{-1}b^{-1}) (ab) = a^{-1}b^{-1} ba = a^{-1} 1 a = 1
\end{align*}
As a result, $a^{-1}b^{-1} = (ab^{-1})$

\noindent \textbf{Ex 2.} Verify the following:
\begin{itemize}
    \item For each real number $a \neq 0$, $a^2 > 0$. In particular, $1 > 0$ since $1 \neq 0$ and $1=1^2$
    \item For each positive number $a$, its multiplicative inverse $a^{-1}$ also is positive
    \item If $a > b$, then
        \begin{equation*}
            ac > bc \text{ if } c > 0 \text{ and } ac < bc \text{ if } c < 0
        \end{equation*}
\end{itemize}
For the first point, we first need to prove that, for any $a$, then $-a=(-1) a$, 
\begin{equation*}
    a + (-a) = 0 = (1 + (-1)) a = a + (-1) a
\end{equation*}
Next, for each $a \neq 0$, if $a$ is positive, then $a^2$ is positive by definition of positiveness. On the other hand, if $a<0$, then let $a = -b$ with $b > 0$, 
\begin{align}
    a^2 = (-b)^2 = (-1)b (-1) b = (-1) (-b) b = (-1)^2 b^2 > 0
\end{align}
For the second point, assuming by contradiction that $a^{-1} < 0$ for any $a>0$, then let $a^{-1} = -b$ with $b>0$. then
\begin{align*}
    1 = a(a^{-1}) = a (-b) = (-1) ab < 0
\end{align*}
Here $ab>0$ since both $a$ and $b$ are positive, and we know from previous point that $0 > -(ab) = (-1) ab$

The last point is straighforward from the definition of $>$.
\begin{align*}
    ac - bc = \underbrace{(a-b)}_{>0}\underbrace{c}_{>0} >0 \\
\end{align*}
\begin{equation*}
    ac - bc = \underbrace{(a-b)}_{>0}\underbrace{c}_{<0} = (-1) \underbrace{(a-b)}_{>0}\underbrace{d}_{>0} < 0 \text{ with } d = -c
\end{equation*}
\textbf{Ex 3.} For a nonempty set of real numbers $E$, show that $\inf E = \sup E$ if and only if $E$ consists of a single point.

If the set $E$ has a single element, then the least upper bound equal to that single element. This similarly applies to lowerbound. In other word, its sup and inf coincides.

On the other direction, if a set $E$ has its sup and inf equal, and assuming by contradiction that $E$ has at least 2 distinct elements, then the gap between these two points $\neq 0$. The difference between sup and inf is lowerbounded by this gap, so they cannot equal.

\noindent \textbf{Ex 4.} Let a