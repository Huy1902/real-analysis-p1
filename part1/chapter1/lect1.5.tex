\newcommand{\seq}[1]{\left\{#1\right\}}
\newcommand{\abs}[1]{\left|#1\right|}

\subsubsection{Summary}

\noindent A sequence is a function $f:\N \to \R$ with customary notation $\seq{a_n}$ 
where $n$ is called the index, the number $a_n$ is the $n$th term.

\noindent A sequence $\seq{a_n}$ is said to be
\begin{itemize}
    \item bounded if $\exists c \geq 0$ s.t. $\abs{a_n} \leq c \forall n$ 
    \item increasing if $a_n < a_{n+1} \forall n$
    \item decreasing if the sequence $\seq{-a_n}$ is increasing
    \item monotone if it's either increasing or decreasing
\end{itemize}

\noindent For any sequence $\seq{a_n}$ and a strictly increase sequence $\seq{n_k} \in \N$,
call the sequence $\seq{a_{n_k}}$ a subsequence of $\seq{a_n}$.

\begin{defn}
    A sequence $\seq{a_n}$ converges to it's limit $a$ (write $\lim_{n \to \infty} a_n = a$ or $\seq{a_n}\to a$) if $\forall \epsilon > 0, \exists N \in \N$ s.t.
    \[ n\geq N \implies \abs{a - a_n} < \epsilon.\]
\end{defn}

\begin{prop}
    If $\seq{a_n}\to a$, then the limit is unique, the sequence is bounded, and, $\forall c \in \R$, 
    \[ a_n \leq c \forall n \implies a \leq c.\]
    Proof \hyperref[ex:1.5.extra]{\underline{\textbf{Ex Extra.}}}
\end{prop}

\begin{thm}
    A monotone sequence of real numbers converges if and only if it is bounded.
\end{thm}
\begin{thm}[Bolzano-Weierstrass]
    Every bounded sequence of real numbers has a convergent subsequence.
\end{thm}
\begin{defn}
    A sequence of real numbers $\seq{a_n}$ is Cauchy if $\forall \epsilon > 0, \exists N\in\N$ s.t.
    \[ n, m \geq N \implies \abs{a_m - a_n} < \epsilon.\]
\end{defn}

\begin{thm}
    A sequence of real numbers converges if and only if it is Cauchy.
\end{thm}
\begin{thm}
    Convergent real sequences are linear and monotonic.
\end{thm}

\begin{defn} 
    A sequence $\seq{a_n}$ converges to infinity (write $\lim_{n \to \infty} a_n = \infty$ or $\seq{a_n}\to \infty$) if $\forall c \in \R, \exists N \in \N$ s.t.
    \[ n\geq N \implies a_n \geq c.\]
    Similar definitions are made at $-\infty$.
\end{defn}

\begin{defn}
    The limit superior and limit inferior of a sequence $\seq{a_n}$ is defined as,
    \begin{align*}
        \limsup\seq{a_n} &= \lim_{n\to\infty} \left[\sup\left\{a_k | k \geq n\right\}\right] \\
        \liminf\seq{a_n} &= \lim_{n\to\infty} \left[\inf\left\{a_k | k \geq n\right\}\right] \\
    \end{align*}
\end{defn}

\begin{prop}
    Let $\seq{a_n}$ and $\seq{b_n}$ be sequences of real numbers
    \begin{enumerate}[label=(\roman*)]
        \item $\limsup\seq{a_n} = \ell \in \R$ if and only if for each $\epsilon > 0$, there are infinitely many indices $n$ for which $a_n > \ell - \epsilon$ and only finitely many indices $n$ for which $a_n > \ell + \epsilon$.
        \item $\limsup\seq{a_n} = \infty$ if and only if $\seq{a_n}$ is not bounded above.
        \item $\limsup\seq{a_n} = -\liminf\seq{-a_n}$.
        \item A sequence of real numbers $\seq{a_n}$ converges to $a \in \R$ if and only if $\liminf\seq{a_n} = \limsup\seq{a_n} = a$.
        \item $a_n \leq b_n \forall n \implies \limsup\seq{a_n} \leq \liminf\seq{b_n}$.
    \end{enumerate}
    Proof \hyperref[ex:39]{\underline{\textbf{Ex 39.}}}
\end{prop}

\begin{defn}
    For every sequence $\seq{a_k}$ of real numbers, define a sequence of partial sums $\seq{s_n}$ where $s_n = \sum_{k=1}^{n}$.
    The series $\sum_{k=1}^{\infty} a_k$ is summable to $s\in\R$ when $\seq{s_n}\to s$.
\end{defn}

\begin{prop}
    Let $\seq{a_n}$ be a sequence of real numbers.
    \begin{enumerate}[label=(\roman*)]
        \item The series $\sum_{k=1}^\infty a_k$ is summable if and only if for each $\epsilon > 0, \exists N\in\N$ s.t.
        \[ \sum_{k=n}^{n+m} a_k < \epsilon \forall n \ge N, m \in \N.\]
        \item If the series $\sum_{k=1}^\infty |a_k|$ is summable, then $\sum_{k=1}^\infty a_k$ also is summable.
        \item If each term $a_k$ is nonnegative, then the series $\sum_{k=1}^\infty a_k$ is summable if and only if the sequence of partial sums is bounded.
    \end{enumerate}
    Proof \hyperref[ex:45]{\underline{\textbf{Ex 45.}}}
\end{prop}

\subsubsection{Excercise}
\underline{\textbf{Problems done: 38, 39, 40, 41, 45. and proved the first Proposition (i.e. Ex Extra.).}}
\bigskip

\noindent \textbf{Ex 38.}
\begin{lem}
    For any set $X\subseteq \R$, $\forall d > 0 \in \R, \exists x \in X$ s.t. $x < \inf X + d$.
\end{lem}

\textbf{Proof} We prove by contradiction. Assume there exists $d > 0 \in \R$ s.t. 
$\forall x \in X, \inf X + d \leq x$. There is now a greater lower bound $\inf X + d$, which contradicts the definition of infimum.

\bigskip

\noindent We use the above lemma to solve this excercise.

\noindent Let $\liminf\seq{a_n} = L$.
\begin{itemize}
    \item $\liminf\seq{a_n}$ is a cluster point.
    
    By the above lemma, for every $n$, we can pick the smallest index $k_n \geq n$ satisfying $a_{k_n} \leq \inf\seq{a_k | k \geq n} + \frac{1}{n}$
    
    Now, $\forall \epsilon > 0, \exists N \in \N, N\geq 1/\epsilon$ s.t. $n \geq N \implies a_{k_n} - L < 1/N \leq \epsilon.$
    The subsequence $\seq{a_{k_n}}$ converges to $L$ by defintion.
    
    \item There does not exist a cluster point $M$ satisfying $M < \liminf\seq{a_n}$.
    
    We argue by contradiction. Assume there exists such a cluster point, this means there also exists a subsequence $\seq{a_{m_j}}$ that converges to $M$.

    Let $\epsilon = \frac{M-L}{2}$, by definition, $\exists J \in \N$ s.t. 
    \[j \geq J \implies a_{m_j} - M < \epsilon \iff a_{m_j} < M + \epsilon = \frac{L+M}{2}.\]
    % In other words, $j\geq J \implies \inf\seq{a_}

    Also, by definition, $L = \liminf\seq{a_n} = \lim_{n\to\infty} \seq{\inf\seq{a_k | k \geq n}}$, as such $\exists N \in \N, N > J$ s.t. 
    \[n \geq N \implies L - \inf\seq{a_k | k \geq n} < \epsilon \iff \inf\seq{a_k | k \geq n} > L - \epsilon = \frac{L + M}{2}.\]

    This is a contradiction, as there exists $N \in \N$ satisfying
    \[ n \geq N \implies \begin{cases}
        a_{m_n} < \frac{L+M}{2} \\ \inf\seq{a_k | k > n} \geq \frac{L+M}{2}
    \end{cases}.\]
\end{itemize}
Proof is similar for $\limsup\seq{a_n}$

\noindent \textbf{Ex 39.} \label{ex:39}
Let $\seq{a_n}$ and $\seq{b_n}$ be sequences of real numbers
\begin{enumerate}[label=(\roman*)]
    \item $\limsup\seq{a_n} = \ell \in \R$ if and only if for each $\epsilon > 0$, there are infinitely many indices $n$ for which $a_n > \ell - \epsilon$ and only finitely many indices $n$ for which $a_n > \ell + \epsilon$.
    
    Trivial. Use definition of suprimum and the fact that the collection of sequences $\seq{\seq{a_k | k \geq n}}_{n=1}^\infty$ is decending.

    \item $\limsup\seq{a_n} = \infty$ if and only if $\seq{a_n}$ is not bounded above.
    
    We prove the above through showing that $\limsup\seq{a_n} < \infty$ if and only if $\seq{a_n}$ is bounded above. Note that the limit superior of a sequence always exists.
    \begin{itemize}
        \item If $\seq{a_n}$ is bounded above, then $\exists M < \infty \in \R$ s.t. $a_n \leq M \forall n$.
        As a result, $\sup\seq{a_k | k \geq n} \leq M$.
        \item If $\limsup\seq{a_n} < \infty$, then $\sup\seq{a_k \ k \geq n}$ is bounded.
        
        And because there exists $c > 0$ satisfying $a_n \leq \sup\seq{a_k | k \geq 1} \leq c$ for all $n$, the sequence $\seq{a_n}$ is also bounded above.
        % \item If $\seq{a_n}$ is not bounded above, then $\forall c, \exists n$ s.t. $a_n \geq c$.
        % And since $\sup\seq{a_k | k \geq n} \geq a_n \forall n$ by definition of suprimum;
        % for any $c\in\R, \exists N\in\N$ s.t. \[ n\geq N \implies \sup\seq{a_k | k\geq n} \geq c.\]
        % By definition, $\lim_{n\to\infty} \sup\seq{a_k | k \geq n} = \limsup\seq{a_n} = \infty$
    \end{itemize}

    \item $\limsup\seq{a_n} = -\liminf\seq{-a_n}$.
    \[\limsup\seq{a_n} = \lim_{n\to\infty}\sup\seq{a_k | k \geq n} = -\lim_{n\to\infty}\inf\seq{-a_k | k \geq n} = -\liminf\seq{-a_n}.\]
    (I ommitted the proof to $\lim_{n\to\infty}\seq{a_n} = -\lim_{n\to\infty}\seq{-a_n}$. It is trivial and uses the definition.)

    \item A sequence of real numbers $\seq{a_n}$ converges to $a \in \R$ if and only if $\liminf\seq{a_n} = \limsup\seq{a_n} = a$.
    
    \begin{itemize}
        \item $\liminf\seq{a_n} = \limsup\seq{a_n} = a \implies \seq{a_n}\to a$
        
        For any $\epsilon > 0$, there exists $N, M\in\N$ s.t.
        \[\begin{cases}
            n\geq N \implies -\epsilon < a - \sup\seq{a_k | k \geq n} \leq a - a_n \\
            n\geq M \implies a - a_n \leq a - \inf\seq{a_k | k \geq n} < \epsilon
        \end{cases} \]

        So $\exists L = \max(N, M) \in \N$ s.t.
        \[ n \geq L \implies \begin{cases}
            -\epsilon < a - a_n \\ a - a_n < \epsilon
        \end{cases} \implies \abs{a - a_n} < \epsilon. \] 

        By definition, $\seq{a_n}\to a$

        \item $\seq{a_n}\to a \implies \liminf\seq{a_n} = \limsup\seq{a_n} = a$
        
        For any $\epsilon > 0$, there exists $N \in\N$ s.t.
        \[\forall n \geq N, \begin{cases}
            -\epsilon < a - a_n \\
            a - a_n < \epsilon
        \end{cases}\implies \forall n \geq N,\begin{cases}
            \inf\seq{a_k | k \geq n} \leq a_n < a + \epsilon \\
            a - \epsilon < a_n \implies a - \epsilon < \inf\seq{a_k | k \geq n} 
        \end{cases}. \]
        which is equivalent to $\abs{a - \inf\seq{a_k | k \geq n}} < \epsilon$ for all $n\geq N$ and so $\liminf\seq{a_n} = a$ by definition.

        Similar proof is done for $\limsup\seq{a_n} = a$.
    \end{itemize}
     

    \item $a_n \leq b_n \forall n \implies \limsup\seq{a_n} \leq \liminf\seq{b_n}$. (similar to book)
    
    Consider the sequence $\seq{c_n}$, where $c_n = \inf\seq{b_k | k \geq n} - \sup\seq{a_k | k \geq n}$ for all $n$. 
    
    By linearity of convergent sequences, $\seq{c_n}\to c = \liminf\seq{b_n} - \limsup\seq{a_n}$.
    This means, $\forall \epsilon > 0, \exists N\in\N$ s.t.
    \[ n\geq N \implies -\epsilon < c - c_n < \epsilon. \]
    In particular, $0\leq c_N < c + \epsilon$. Since $c \geq -\epsilon$ for any positive number $\epsilon$, $c \geq 0$.
\end{enumerate}

\noindent \textbf{Ex 40.}

Proven above in \underline{\textbf{Ex. 38}}, $\liminf\seq{a_n}$ and $\limsup\seq{a_n}$ are the smallest and largest cluster points of $\seq{a_n}$.

Shown above in \underline{\textbf{Ex. 39}}, $\seq{a_n}\to a \iff \liminf\seq{a_n} = \limsup\seq{a_n} = a$.

The proof is now trivial.

The sequence $\seq{a_n}$ has only one cluster point if and only if $\liminf\seq{a_n} = \limsup\seq{a_n} = a$, which is equivalent to $\seq{a_n} \to a$.

\noindent \textbf{Ex 41.}
% Denote the sequences $\seq{i_n}, \seq{s_n}$ where $i_n = \inf\seq{a_k | k \geq n}$ and $s_n = \sup\seq{a_k | k \geq n}$
At every index $n$,
\[ \inf\seq{a_k | k \geq n} \leq \sup\seq{a_k | k \geq n} \]

And so, by the linearity property of convergent sequences, $\lim_{n\to\infty}\inf\seq{a_k | k \geq n} \leq \lim_{n\to\infty}\sup\seq{a_k | k \geq n}$ or $\liminf\seq{a_n} \leq \limsup\seq{a_n}$.

\noindent \textbf{Ex 43.} \label{ex:43} Show that every real sequence has a monotone subsequence. Use this to provide another proof of the Bolzano-Weierstrass Theorem. 

By contradiction, we assume that there exists a sequence $\{x_i\}_i$ such that it has no monotone subsequence. We devide the proof into two parts: a squence that has no increasing subsequence has a decreasing subsequence and vice versa. We only prove the first part since the other part are identical.

Note that we use the terminology similar to what is defined in the textbook: an increasing sequence has $a_n \leq a_{n+1}$. This is a bit misleading as a more correct term would be "non-decreasing" sequence. 

First, assuming that the sequence $\{x_i\}$ has no increasing subsequence; that is: every construstion of such subsequences stops at some finite steps. More specifically, 

\begin{itemize}
    \item Starting at $x_0$, we construct the longest increasing subsequence ${x_0, \dots x_m}$. This sequence is finite due to our assumption that $\{x_i\}$ has no increasing subsequence. 
    \item We then have $x_m>x_j, \forall j > m$ or otherwise we can concatenate $x_j$ to the subsequence found in the previous step to make a longer sequence.
    \item Let $a_m = x_m$, repeat this process with the rest of sequence $\{x_i\}_{i>m}$, the sequence $\{a_m\}$ is a decreasing sequence. 
\end{itemize}

\noindent \textbf{Ex 45.} \label{ex:45}
Let $\seq{a_n}$ be a sequence of real numbers.
\begin{enumerate}[label=(\roman*)]
    \item The series $\sum_{k=1}^\infty a_k$ is summable if and only if for each $\epsilon > 0, \exists N\in\N$ s.t.
    \[ \abs{\sum_{k=n}^{n+m} a_k} < \epsilon \forall n \geq N, m \in \N.\]
    
    The series $\sum_{k=1}^{\infty} a_k$ is summable if and only if $\seq{s_n}$ converges.

    As such, for each $\epsilon > 0, \exists N \in \N$ s.t.
    \begin{align*}
        j > i-1 \geq N &\implies \epsilon > \abs{\sum_{k=i}^{j} a_k} \\
        \iff n \geq N, m \in \N &\implies \epsilon > \abs{\sum_{k=n}^{n+m} a_k} \quad (i-1 = n, j = n+m)
    \end{align*}
    
    \item If the series $\sum_{k=1}^\infty \abs{a_k}$ is summable, then $\sum_{k=1}^\infty a_k$ also is summable.

    If the series $\sum_{k=1}^\infty \abs{a_k}$ is summable, then the partial sum sequence $\seq{\sum_{k=1}^{n} \abs{a_k}}$ converges.
    
    As such, $\forall \epsilon > 0, \exists N \in\N$ s.t.
    \[ n,m \geq N \implies \epsilon > \abs{\sum_{k=\min(m,n)}^{\max(m,n)} \abs{a_k}} \geq \abs{\sum_{k=\min(m,n)}^{\max(m,n)} a_k}. \]
    The partial sum sequence $\seq{\sum_{k=1}^{n} a_k}$ converges because it is Cauchy. As a result, the series $\sum_{k=1}^\infty a_k$ also is summable.

    \item If each term $a_k$ is nonnegative, then the series $\sum_{k=1}^\infty a_k$ is summable if and only if the sequence of partial sums is bounded.
    
    Since $a_k > 0 \forall k\in\N$, $s_n = \sum_{k=1}^{n} a_k \leq \sum_{k=1}^{n+1} a_k = s_{n+1}$ for all $n$. In other words, the partial sum sequence is nondecreasing.

    The series $\sum_{k=1}^{\infty} a_k$ is summable if and only if $\seq{s_n}$ converges.
    \begin{itemize}
        \item If $\seq{s_n}$ converges then it is bounded.
        \item If $\seq{s_n}$ is bounded, then it converges to $s = \sup\seq{s_n | n \in \N}$ (note that the suprimum exists thanks to the Completeness Axiom)
        
        For any $\epsilon > 0$, we have:
        \begin{itemize}[label=+)]
            \item $s_n \leq s < s + \epsilon$ for all $n$.
            \item Because $s - \epsilon$ is not an upperbound of $\seq{s_n | n\in\N}$, $\exists N\in\N$ s.t $s_N > s-\epsilon$.
            
            And since the sequence $\seq{s_n}$ is nondecreasing, $n\geq N \implies s_n > s-\epsilon$.
        \end{itemize}
        By definition, $\seq{s_n}$ converges to $s$.
    \end{itemize}
\end{enumerate}

\noindent \textbf{Ex Extra.} \label{ex:1.5.extra}
If $\seq{a_n}\to a$, then:
\begin{itemize}
    \item The limit is unique.
    
    We prove by contradiction. Assume $\seq{a_n}\to a, \seq{a_n}\to b$ and $a\neq b$.
    
    Let $d = \abs{a-b}$ and $\epsilon = \frac{d}{2}$.
    By definition, there exists $N, M\in\N$ s.t.
    \[\begin{cases}
        n\geq N \implies \abs{a - a_n} < \epsilon \\
        n\geq M \implies \abs{b - a_n} < \epsilon
    \end{cases} \]

    So $\exists L = \max(N, M) \in \N$ s.t. $n \geq L$ implies both $\abs{a - a_n}$ and $\abs{b - a_n}$ are less than $\epsilon$.
    
    By the triangle inequality, $d = \abs{a-b} \leq \abs{a - a_n} + \abs{b - a_n} < 2\epsilon = 2\times\frac{d}{2} = d$. In other words, $d < d$, which is a contradiction.

    \item The sequence is bounded.
    
    Choose any $\epsilon > 0$.
    
    By definition, $\exists N\in\N$ s.t.
    \[ n\geq N \implies -\epsilon < a-a_n < \epsilon \iff a-\epsilon < a_n < a+\epsilon \implies \abs{a_n} < \abs{a} + \epsilon \]

    Denote $M_1 = \max\left[\seq{a_n | n\in\N, n < N}\right]$, note that we can always find $M_1$ because this sequence is finite.
    
    We conclude that $\seq{a_n}$ is bounded by $\max\left(\abs{a} + \epsilon, M_1\right)$.

    \item $\forall c \in \R$, if $a_n \leq c \forall n$ then $a \leq c$.
    
    \begin{enumerate}[label=\underline{Approach \arabic*)}]
        \item Using only the definition.
        
        For any $\epsilon > 0, \exists N\in\N$ s.t
        \[ n \geq N \implies \abs{a - a_n} < \epsilon \implies a-\epsilon < a_n \leq c\]

        Since $a-\epsilon < c$ is true for all $\epsilon > 0$, we conclude that $a \leq c$.
        
        \item Using only the definition.
        
        Prove by contradiction. Assume $a > c$, then set $\epsilon = a - c > 0$...
        
        \item Consider the sequence $\seq{c_n}$, where $c_n = c \forall n$ and use the monotonic property of convergent sequences. (Trivial)
    \end{enumerate}
\end{itemize}
